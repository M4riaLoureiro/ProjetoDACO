\documentclass{article}
\usepackage[utf8]{inputenc}
\usepackage[T1]{fontenc}
\usepackage[font=footnotesize]{caption}
\usepackage{fancyhdr}
\usepackage[document]{ragged2e}
\usepackage[super]{nth}
\usepackage{parskip}


\pagestyle{headings}
\setlength{\parindent}{0pt}

\textwidth = 400pt
\oddsidemargin = 35pt
\topmargin = -20pt
\textheight = 650pt

\title{\textbf{Project Proposal}\par Computer-Aided Diagnostics}
\date{\nth{10} November 2020}

\begin{document}

\maketitle

\centering
\textbf{Students}\par
João Carlos Ramos Gonçalves Matos – up201704111\par
Maria Jorge Miranda Loureiro – up201704188\par
Maria Manuel Domingos Carvalho – up201706990\par

\vspace{5mm}
\justify
\setlength{\parindent}{0pt}

\textbf{Dataset - BVC Voice Dataset [1]}\par

The Biometrics Visions and Computing (BVC) Gender & Age from voice dataset comprises voice utterances from 526 individuals at one to five voice recordings per individual, of which 336 are males and 190 are females. The total number of voice utterances are 3,964 consisting of 2,149 male and 1,815 female voice utterances. Five different speeches of English and the equivalent translated native languages were acquired from the subjects in the first and second sessions. One to five voice recordings in English language and Native languages were acquired from each of the subjects. Twenty-eight different native languages make up the native language set.\par


\vspace{5mm}
\textbf{Project Idea}\par
Voice verification and gender classification from voice will be carried out in the context of native (mother tongue) languages and lingua franca languages.\par

\vspace{5mm}
\textbf{Software required}\par
Python

\vspace{5mm}
\textbf{Papers to read (1-3)}\par
[1]	O. Iloanusi et al., Voice Recognition and Gender Classification in the Context of Native Languages and Lingua Franca. 2019, pp. 175-179.




\end{document}


